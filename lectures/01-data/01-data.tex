\documentclass[spanish, c]{beamer}

\usepackage[utf8]{inputenc}
%\usepackage[spanish, mexico]{babel}
\usepackage{amsmath}
\usepackage{mathtools}
\usepackage{hyperref}
\usepackage{xcolor}
\usepackage{color}
\usepackage{ragged2e}
\usepackage{mathrsfs}
% \usepackage{csquotes}
% \usepackage{listings}
\usepackage[scaled]{beramono}
\usepackage[T1]{fontenc}
\usepackage{graphicx}
\usepackage{booktabs}
\usepackage{physics}
\usepackage{minted}
\usepackage{relsize}

\renewcommand{\indent}{\hspace*{2em}}

\newcommand\CC{C\nolinebreak[4]\hspace{-.05em}\raisebox{.4ex}{\relsize{-3}{\textbf{++}}}}

% \usepackage{tikz}

% \usetikzlibrary{fit, shapes, arrows}

% \usepackage{courier}
% \usepackage{subfigure}
% \usepackage{enumerate}
% \usepackage{algorithmic}
% \usepackage{algorithm}

% \usepackage{listings}
% \usepackage{lstlinebgrd}

\usetheme{Boadilla}
\usefonttheme[onlymath]{serif}

\newcommand\blfootnote[1]{%
\begingroup
\renewcommand\thefootnote{}\footnote{#1}%
\addtocounter{footnote}{-1}%
\endgroup
}

% \lstset
% {
%     language = C++,
%     basicstyle = \scriptsize,
%     escapechar = `,
%     numbers = left,
%     frame = tb,
% }

% \lstdefinestyle{output}
% {
%     language = {},
%     basicstyle = \scriptsize,
%     escapechar = `,
%     numbers = none,
%     showtabs = false,
%    	showstringspaces = false,
% }

% Sets the templates
\definecolor{navyblue}{RGB}{0, 0, 128}
\definecolor{crimson}{RGB}{128, 16, 0}

\setbeamertemplate{navigation symbols}{}
\setbeamertemplate{headline}{}
\setbeamertemplate{title page}[default][colsep=-4bp,rounded=true]
\setbeamertemplate{footline}[frame number]
\setbeamertemplate{bibliography item}[text]
\setbeamertemplate{theorems}[numbered]

\setbeamercolor{title}{fg=navyblue, bg=white}
\setbeamercolor{frametitle}{fg=navyblue, bg=white}
\setbeamercolor{structure}{fg=navyblue}
\setbeamercolor{button}{fg=white,bg=navyblue}

\setbeamercovered{transparent}

\title{\textit{Programming Primer}\\Datos, abstracción y matemáticas \texorpdfstring{\vspace{2.5ex}}{}}
\subtitle{Programación de Estructuras de Datos y Algoritmos Fundamentales \\ (TC1031)}
\author{
    \texorpdfstring{
        \begin{center}
            M.C. Xavier Sánchez Díaz \\
            \href{mailto:mail@tec.mx}{\texttt{mail@tec.mx}}
        \end{center}
    }
    {M.C. Xavier Sánchez Díaz}
}

\institute[Tecnológico de Monterrey]{\includegraphics[scale=0.5]{../img/logo}}
\date{}

\begin{document}

\setlength{\rightskip}{0pt}

\begin{frame}[plain]
    \titlepage        
\end{frame}

\begin{frame}{Outline}
    \tableofcontents
\end{frame}

\section{Vocabulario}

\begin{frame}{Términos comunes}{Conceptos básicos de programación}
    \begin{columns}
        \begin{column}{0.33\textwidth}
            \begin{itemize}
                \item Estatuto (\textit{Statement})
                \item Función (\textit{Function})
                \item Rutina (\textit{Routine})
                \item Procedimiento (\textit{Procedure})
            \end{itemize}
        \end{column}
        \begin{column}{0.33\textwidth}
            \begin{itemize}
                \item Parámetro (\textit{Parameter})
                \item Argumento (\textit{Argument})
                \item Tipo (\textit{Type})
                \item Retorno (\textit{Return})
            \end{itemize}
        \end{column}
        \begin{column}{0.3\textwidth}
            \begin{itemize}
                \item Estructura (\textit{Structure})
                \item Arreglo (\textit{Array})
                \item Objeto (\textit{Object})
                \item Bucle (\textit{Loop})
            \end{itemize}
        \end{column}
    \end{columns}
    \begin{columns}
        \begin{column}{0.33\textwidth}
            \begin{itemize}
                \item Condicional (\textit{Conditional})
                \item Palabra reservada (\textit{Reserved keyword})
                \item Archivo (\textit{File})
                \item Directorio (\textit{Directory})
            \end{itemize}
        \end{column}
        \begin{column}{0.33\textwidth}
            \begin{itemize}
                \item Clase (\textit{Class})
                \item Operador (\textit{Operator})
                \item Iteración (\textit{Iteration})
                \item Variable (\textit{Variable})
            \end{itemize}
        \end{column}
        \begin{column}{0.33\textwidth}
            \begin{itemize}
                \item Entero (\textit{Integer})
                \item Punto Flotante (\textit{Floating-point})
                \item Cadena de caracteres (\textit{String})
                \item Imprimir (\textit{Print})
            \end{itemize}
        \end{column}
    \end{columns}
\end{frame}

\section{Datos}

\begin{frame}{Datos como resultados}{Datos}
    Antes de usar la computadora o la calculadora para hacer cálculos, solíamos hacer las operaciones a mano.

    \bigskip
    
    Por ejemplo, si queremos calcular $1270 \times 35$, una manera de hacerlo podría ser\dots

\end{frame}

\begin{frame}{Datos como resultados}{Datos}
    \begin{align*}
        1270 \times 35 & = \\ \pause
        & = (1200 + 70) \times (7)(5) \\ \pause
        & = (\alert<4>{12})(\alert<5,6>{7})(\alert<4>{5})(\alert<7>{100}) + (\alert<8>{7})(\alert<8>{7})(\alert<10>{5})(\alert<9>{10}) \pause
    \end{align*}
    \vspace{-2ex}
    \begin{align*}
        \alert<4>{12} \times \alert<4>{5} = \alert<5,6>{60} \\
        \alert<5,6>{60} \times \alert<5,6>{7} = \alert<6>{6} \times \alert<6>{7} \times \alert<6>{10} = \alert<7>{420} \\
        \alert<7>{420} \times \alert<7>{100} = \alert<11>{42000} \\[1.5ex]
        \alert<8>{7} \times \alert<8>{7} = \alert<9>{49} \\
        \alert<9>{49} \times \alert<9>{10} = \alert<10>{490} \\
        \alert<10>{490} \times \alert<10>{5} = \alert<10>{490} \times \alert<10>{10 / 2} = \alert<11>{2450}\\[1.5ex]
        \alert<11>{42000} + \alert<11>{2450} = 44450 \quad \square
    \end{align*}
\end{frame}

\begin{frame}{Datos como resultados}{Datos}    
    La operación completa se hace poco a poco, y por tanto necesitamos ``recordar'' ciertos pasos intermedios que ya tenemos calculados. \pause

    \bigskip
    
    Así como nosotros tenemos que tener en claro cuáles son esos pasos intermedios, la computadora debe saber \textit{dónde está} la información que tiene que leer para trabajar y hacer cálculos más elaborados. \pause

    \bigskip

    Para eso, podemos usar las estructuras de datos, para \textbf{ordenarlos} de manera conveniente y poder tener acceso a ellos de manera que se vayan necesitando.

\end{frame}

\begin{frame}{Datos y errores}{Datos}
    ¿Qué habría pasado si hubiera muchos puntos decimales de por medio? \pause

    \bigskip

    ¿Cuál es la representación decimal de $\dfrac{1}{7}$? ¿Y de $\dfrac{2}{7}$? \pause

    \bigskip

    \begin{center}
        \LARGE
        \textit{Story time: The Wolf}
    \end{center}    
\end{frame}

\begin{frame}{Tipos de datos}{Datos}
    
    Existen distintos \alert{tipos de datos} con los que podemos trabajar en una computadora: \pause
 
     \bigskip
 
     \begin{itemize}[<+->]
         \item Números enteros (\textit{integer numbers})
         \item Números decimales (\textit{floating point numbers})
         \item Cadenas de caracteres alfanuméricos (\textit{strings})
         \item Datos estructurados definidos por nosotros mismos
     \end{itemize} \pause
 
     \bigskip
 
     Cuando \textbf{operamos} con estos datos, generamos nueva información que podríamos necesitar en el futuro.
 
 \end{frame}

\section{Abstracción de datos}

\begin{frame}{¿Qué es un dato abstracto?}{Abstracción de datos}
    \begin{center}
        \huge
     Un \alert{dato abstracto} es una manera \textit{fancy} de decirle a una \textbf{estructura de datos} definida por el usuario.
    \end{center}
\end{frame}

\begin{frame}{Arreglos}{Abstracción de datos}
    Asumamos que quiero saber las calificaciones de las Tareas 1, 2 y 3 de uno de mis alumnos.
    Para esto, necesitaría un lugar para guardar esos \textbf{3 datos}: \pause

    \bigskip

    \begin{itemize}[<+->]
        \item $t_1 = 90$ será la variable para la Tarea 1
        \item $t_2 = 75$ será la variable para la Tarea 2
        \item $t_3 = 87$ será la variable para la Tarea 3
    \end{itemize}

    \bigskip

    Con esta información, ahora contesta: \pause

    \begin{itemize}[<+->]
        \item ¿Cuál fue la calificación para la Tarea 2?
        \item ¿Cuál fue el promedio del alumno?
        \item ¿Cuál es la tarea en la que mejor le fue?
    \end{itemize}
\end{frame}

\begin{frame}{Arreglos}{Abstracción de Datos}
    La pregunta ahora es\dots ¿realmente necesito \textbf{3 variables} para guardar \textbf{3 datos}?
    Podemos \textit{arreglar} los datos de tal manera que \textbf{su posición} nos aporte algo más: \pause

    $$\mathbf{t} = \langle 90, 75, 87 \rangle$$ \pause

    La \textbf{posición} en esta \textit{estructura} nos indica qué número de tarea fue, y el valor que haya en dicha posición guarda la calificación. Por lo mismo, podemos usar ``una sola variable'' para guardar de manera estructurada la información requerida, y referirnos sólo a la posición deseada:

    $$\mathbf{t}_2 = 75$$
\end{frame}

\begin{frame}{Abstracción}{Abstracción de Datos}

    \begin{center}
        \LARGE
        \textit{Story time: La Torre Eiffel}
    \end{center} \pause

    \begin{center}
        \LARGE
        \textit{Story time: La presentación del equipo 5}
    \end{center}
\end{frame}

\begin{frame}{Niveles de abstracción}{Abstracción de Datos}
    Independientemente de la \textit{estructura} de nuestros datos, a nivel abstracto siempre trabajamos con un \alert{objeto}.\blfootnote{Fun fact: cambia el sustantivo ``objeto'' por \textit{cosa {\tiny(o madre)}} para todo en esta slide y seguro lo recuerdas mejor} \pause
    \
    \bigskip

    \begin{itemize}[<+->]
        \item Este \textbf<5->{objeto} tiene varias \textbf<5->{cosas}.
        \item Este \textbf<5->{objeto} me sirve para \textbf<5->{otra cosa}.
        \item Este \textbf<5->{objeto} puedo usarlo de \textbf<5->{esta manera}.
    \end{itemize}

    \pause

    \bigskip

    Esto es \alert{abstracción}.
\end{frame}

\begin{frame}{Niveles de abstracción}{Abstracción de Datos}
    
    \begin{itemize}[<+->]
        \itemsep2.5ex
        \item Una \alert{clase} es un \textit{tipo} de \textbf{objeto}
        \item Una \textbf{clase} tiene \alert{atributos}
        \item Una \textbf{clase} tiene \textit{operaciones} que puede realizar sobre sus \textbf{atributos}. Estas operaciones son \alert{funciones} específicamente de la clase
        \item A estas \textbf{funciones} de la clase se les llama \alert{métodos}
        \item Tanto los \textbf{métodos} como los \textbf{atributos} pueden ser \alert{privados} o \alert{públicos} dependiendo quién deba tener acceso a esa información.
    \end{itemize}
\end{frame}

\section{Vocabulario avanzado}

\begin{frame}{Términos comunes}{Conceptos básicos de programación orientada a objetos}
    \begin{columns}
        \begin{column}{0.33\textwidth}
            \begin{itemize}
                \item Clase (\textit{Class})
                \item Objeto (\textit{Object})
                \item Clase abstracta (\textit{Abstract Class})
                \item Interfaz (\textit{Interface})
                \item Instancia (\textit{Instance})
                \item Atributo (\textit{Attribute})
            \end{itemize}
        \end{column}
        \begin{column}{0.33\textwidth}
            \begin{itemize}
                \item Método (\textit{Method})
                \item Plantilla o Templete (\textit{Template})
                \item Encapsulación (\textit{Encapsulation})
                \item Polimorfismo (\textit{Polymorphism})
                \item Herencia (\textit{Inheritance})
            \end{itemize}
        \end{column}
        \begin{column}{0.3\textwidth}
            \begin{itemize}
                \item Cohesión (\textit{Cohesion})
                \item Acoplamiento (\textit{Coupling})
                \item \textit{DRY}
                \item \textit{A function should do one thing}
            \end{itemize}
        \end{column}
    \end{columns}
\end{frame}

\begin{frame}[fragile]{Ejemplo en \texorpdfstring{\protect\CC}{C++}}{Pa' recordarlo\dots}
    \inputminted[linenos, breaklines, fontsize=\scriptsize]{c++}{../../hello.cpp}
\end{frame}

% \section*{Referencias}

% \begin{frame}[t]{Referencias}
    % \nocite{bibID01}
    % \nocite{bibID02}

    % \bibliographystyle{IEEE}
    % \bibliography{biblio}
% \end{frame}

\end{document}