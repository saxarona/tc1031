\documentclass[spanish, c]{beamer}

\usepackage[utf8]{inputenc}
%\usepackage[spanish, mexico]{babel}
\usepackage{amsmath}
\usepackage{mathtools}
\usepackage{hyperref}
\usepackage{xcolor}
\usepackage{color}
\usepackage{ragged2e}
\usepackage{mathrsfs}
% \usepackage{csquotes}
% \usepackage{listings}
\usepackage[scaled]{beramono}
\usepackage[T1]{fontenc}
\usepackage{graphicx}
\usepackage{booktabs}
\usepackage{physics}
\usepackage{minted}
\usepackage{tcolorbox}
\usepackage{tikz}
\usepackage{relsize}

\tcbuselibrary{minted, skins}

\renewcommand{\indent}{\hspace*{2em}}

\newcommand\CC{C\nolinebreak[4]\hspace{-.05em}\raisebox{.4ex}{\relsize{-3}{\textbf{++}}}~}
% \usepackage{tikz}

% \usetikzlibrary{fit, shapes, arrows}

% \usepackage{courier}
% \usepackage{subfigure}
% \usepackage{enumerate}
% \usepackage{algorithmic}
% \usepackage{algorithm}

% \usepackage{listings}
% \usepackage{lstlinebgrd}

\usetheme{Boadilla}
\usefonttheme[onlymath]{serif}

\newcommand\blfootnote[1]{%
\begingroup
\renewcommand\thefootnote{}\footnote{#1}%
\addtocounter{footnote}{-1}%
\endgroup
}

% \lstset
% {
%     language = C++,
%     basicstyle = \scriptsize,
%     escapechar = `,
%     numbers = left,
%     frame = tb,
% }

% \lstdefinestyle{output}
% {
%     language = {},
%     basicstyle = \scriptsize,
%     escapechar = `,
%     numbers = none,
%     showtabs = false,
%    	showstringspaces = false,
% }

% Sets the templates
\definecolor{navyblue}{RGB}{0, 0, 128}
\definecolor{crimson}{RGB}{128, 16, 0}

\setbeamertemplate{navigation symbols}{}
\setbeamertemplate{headline}{}
\setbeamertemplate{title page}[default][colsep=-4bp,rounded=true]
\setbeamertemplate{footline}[frame number]
\setbeamertemplate{bibliography item}[text]
\setbeamertemplate{theorems}[numbered]

\setbeamercolor{title}{fg=navyblue, bg=white}
\setbeamercolor{frametitle}{fg=navyblue, bg=white}
\setbeamercolor{structure}{fg=navyblue}
\setbeamercolor{button}{fg=white,bg=navyblue}

\setbeamercovered{transparent}

\tcbset{cppexample/.style={%
    colback=green!5,
    colframe=green!30!black,
    listing only,
    fonttitle=\bfseries,
    listing engine=minted,
    minted language=c++,
    minted options={fontsize=\scriptsize, breaklines, linenos, autogobble, numbersep=2mm},
    enhanced,
    overlay={\begin{tcbclipinterior}\fill[red!25!green!25!white] (frame.south west)rectangle ([xshift=4mm]frame.north west);\end{tcbclipinterior}}
}}

\tcbset{cppfullexample/.style={%
    % colback=green!5,
    % colframe=green!30!black,
    listing only,
    fonttitle=\bfseries,
    listing engine=minted,
    minted language=c++,
    minted options={fontsize=\scriptsize, breaklines, linenos, autogobble, numbersep=2mm},
    enhanced,
    overlay={\begin{tcbclipinterior}\fill[black!20!white] (frame.south west)rectangle ([xshift=4mm]frame.north west);\end{tcbclipinterior}}
}}

\tcbset{cppfullborderless/.style={%
    % colback=green!5,
    % colframe=green!30!black,
    listing only,
    listing engine=minted,
    minted language=c++,
    minted options={fontsize=\scriptsize, breaklines, linenos, autogobble, numbersep=2mm},
    enhanced,
    overlay={\begin{tcbclipinterior}\fill[black!20!white] (frame.south west)rectangle ([xshift=4mm]frame.north west);\end{tcbclipinterior}}
}}

\title{\textit{Programming Primer II}\\Templates, Iteración y Recursión \texorpdfstring{\vspace{2.5ex}}{}}
\subtitle{Programación de Estructuras de Datos y Algoritmos Fundamentales \\ (TC1031)}
\author{
    \texorpdfstring{
        \begin{center}
            M.C. Xavier Sánchez Díaz \\
            \href{mailto:mail@tec.mx}{\texttt{mail@tec.mx}}
        \end{center}
    }
    {M.C. Xavier Sánchez Díaz}
}

\institute[Tecnológico de Monterrey]{\includegraphics[scale=0.5]{../img/logo}}
\date{}

\begin{document}

\setlength{\rightskip}{0pt}

\begin{frame}[plain]
    \titlepage        
\end{frame}

\begin{frame}{Outline}
    \tableofcontents
\end{frame}

\section{Templates}

\begin{frame}[fragile]{Data Types en \CC}{Templates}
    \CC es un lenguaje \textit{fuertemente tipado}---funciones, objetos y estructuras deben tener un tipo de dato asociado al compilar y es el único tipo de datos que aceptarán.

    \bigskip

    \begin{tcblisting}{cppexample, title={Boolean return with integer parameters}}
        bool greaterThanInt(int a, int b){
            bool g = false;
            if (a > b){
                g =  true;
            }
            return g;
        }
    \end{tcblisting}

\end{frame}

\newtcbinputlisting{\fullExampleA}[2][]{%
    listing file={#2},
    cppfullexample,
    title={Boolean return with double precision floating-point params}
}

\begin{frame}[fragile, plain]{Data Types en \CC}{Templates}
    \fullExampleA[before upper=\textit{File contents:}]{approxEqual.cpp}
\end{frame}

\begin{frame}[fragile, allowframebreaks]{Templates en \CC}{Templates}
    Podemos convertir una \textit{función tipada} en un \alert{\textit{template}} para poder usar múltiples tipos de datos al incluir el decorador \mintinline{c++}{template <class mytype>} justo arriba de una \textbf{clase} o \textbf{función} para hacerla \textit{genérica}:
    \vspace{2ex}
    \begin{tcblisting}{cppexample, title={Defining a template}, before upper=\scriptsize\texttt{Lista.h}}
        #define MAX 100
        template <class T>
        class Lista
        {
        private:
            T data[MAX];
            int size;
        public:
            Lista();
        ...
    \end{tcblisting}

    Después, durante la implementación, especificamos el tipo de dato específico a utilizar, por ejemplo:
    
    \bigskip

    \begin{tcblisting}{cppexample, title={Implementing a template}, before upper=\scriptsize\texttt{miLista.cpp}}
        #include "Lista.h"
        int main(){
            Lista<float> miLista;
            ...
    \end{tcblisting}

    para implementar una lista que guarde solamente números decimales.
\end{frame}

\section{Patrones de diseño de algoritmos}

\begin{frame}{Patrones de diseño}{Iteración}
    
    En el ámbito de software, llamamos \alert{patrones de diseño} a la manera en la que se `resuelve' un problema. Existen muchos patrones de diseño para algoritmos, pero en el curso revisaremos dos enfoques importantes: \pause

    \bigskip

    \begin{itemize}
        \item \textbf{Métodos Iterativos}. Se usan para recorrer contenedores con estructuras indizables. \pause
        \item \textbf{Métodos Recursivos}. Se usan para recorrer o generar contenedores con estructuras recursivas \pause
    \end{itemize}
    
    \bigskip

    Cada uno tiene sus propiedades y tienen fundamentos matemáticos distintos.
\end{frame}

\begin{frame}{Iteración}{Patrones de diseño de algoritmos}
    La \alert{iteración} está basada en las \textbf{estructuras matemáticas ordenadas} como las tuplas, las cadenas de caracteres y los arreglos $n$-dimensionales. \pause

    \bigskip

    $$\langle \alert<4>{1}, \alert<5>{2}, \alert<6>3, \dots \rangle$$ \pause

    $$\begin{bmatrix}
        \alert<7>{2} & \alert<8>{4} \\
        \alert<9>{8} & \alert<10>{16} \\
        \alert<11>{32} & \alert<12>{64}
    \end{bmatrix}$$

\end{frame}

\begin{frame}{Recursión}{Patrones de diseño de algoritmos}
    La \alert{recursión} es una propiedad estructural de algunos objetos matemáticos (como los números naturales $\mathbb{N}$) que se puede simplificar en dos pasos: \pause

    \begin{enumerate}
        \item Existe un caso base \pause
        \item Existe una regla general basada en el caso anterior para todos los siguientes casos \pause
    \end{enumerate}

    \begin{exampleblock}{Inducción en $\mathbb{N}$}
        \begin{itemize}
            \item \textbf{Caso base:} 1 \pause
            \item \textbf{Regla general:} sumar 1 al número anterior \pause
        \end{itemize}

        Con esto podemos generar \textbf{todos} los números naturales ($1, 2, 3, 4, \dots$):

        \[{\color{blue} 1}, ({\color{blue} 1}) + {\color{cyan} 1}, (({\color{blue}1}) + {\color{cyan}1}) + {\color{magenta}1}, ((({\color{blue}1}) + {\color{cyan}1}) + {\color{magenta}1}) + 1, \dots\]
    \end{exampleblock}
\end{frame}

\newtcbinputlisting{\fullExampleB}[2][]{%
    listing file={#2},
    cppfullborderless
}

\newtcbinputlisting{\fullExampleC}[2][]{%
    listing file={#2},
    cppfullborderless
}

\begin{frame}[fragile, plain]{Factorial}{Iteración}
    \fullExampleB[before upper=\textit{File contents:}]{factorialIter.cpp}
\end{frame}

\begin{frame}[fragile, plain]{Factorial}{Recursión}
    \fullExampleC[before upper=\textit{File contents:}]{factorialRecur.cpp}
\end{frame}

% \section*{Referencias}

% \begin{frame}[t]{Referencias}
    % \nocite{bibID01}
    % \nocite{bibID02}

    % \bibliographystyle{IEEE}
    % \bibliography{biblio}
% \end{frame}

\end{document}